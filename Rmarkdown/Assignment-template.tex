\documentclass[12pt,english,]{article}
\usepackage{lmodern}
\usepackage{amssymb,amsmath}
\usepackage{ifxetex,ifluatex}
\usepackage{fixltx2e} % provides \textsubscript
\ifnum 0\ifxetex 1\fi\ifluatex 1\fi=0 % if pdftex
  \usepackage[T1]{fontenc}
  \usepackage[utf8]{inputenc}
\else % if luatex or xelatex
  \ifxetex
    \usepackage{mathspec}
  \else
    \usepackage{fontspec}
  \fi
  \defaultfontfeatures{Ligatures=TeX,Scale=MatchLowercase}
\fi
% use upquote if available, for straight quotes in verbatim environments
\IfFileExists{upquote.sty}{\usepackage{upquote}}{}
% use microtype if available
\IfFileExists{microtype.sty}{%
\usepackage{microtype}
\UseMicrotypeSet[protrusion]{basicmath} % disable protrusion for tt fonts
}{}
\usepackage[margin=1in]{geometry}
\usepackage{hyperref}
\hypersetup{unicode=true,
            pdfborder={0 0 0},
            breaklinks=true}
\urlstyle{same}  % don't use monospace font for urls
\ifnum 0\ifxetex 1\fi\ifluatex 1\fi=0 % if pdftex
  \usepackage[shorthands=off,main=english]{babel}
\else
  \usepackage{polyglossia}
  \setmainlanguage[]{english}
\fi
\usepackage{graphicx}
% grffile has become a legacy package: https://ctan.org/pkg/grffile
\IfFileExists{grffile.sty}{%
\usepackage{grffile}
}{}
\makeatletter
\def\maxwidth{\ifdim\Gin@nat@width>\linewidth\linewidth\else\Gin@nat@width\fi}
\def\maxheight{\ifdim\Gin@nat@height>\textheight\textheight\else\Gin@nat@height\fi}
\makeatother
% Scale images if necessary, so that they will not overflow the page
% margins by default, and it is still possible to overwrite the defaults
% using explicit options in \includegraphics[width, height, ...]{}
\setkeys{Gin}{width=\maxwidth,height=\maxheight,keepaspectratio}
\IfFileExists{parskip.sty}{%
\usepackage{parskip}
}{% else
\setlength{\parindent}{0pt}
\setlength{\parskip}{6pt plus 2pt minus 1pt}
}
\setlength{\emergencystretch}{3em}  % prevent overfull lines
\providecommand{\tightlist}{%
  \setlength{\itemsep}{0pt}\setlength{\parskip}{0pt}}
\setcounter{secnumdepth}{0}
% Redefines (sub)paragraphs to behave more like sections
\ifx\paragraph\undefined\else
\let\oldparagraph\paragraph
\renewcommand{\paragraph}[1]{\oldparagraph{#1}\mbox{}}
\fi
\ifx\subparagraph\undefined\else
\let\oldsubparagraph\subparagraph
\renewcommand{\subparagraph}[1]{\oldsubparagraph{#1}\mbox{}}
\fi

%%% Use protect on footnotes to avoid problems with footnotes in titles
\let\rmarkdownfootnote\footnote%
\def\footnote{\protect\rmarkdownfootnote}

%%% Change title format to be more compact
\usepackage{titling}

% Create subtitle command for use in maketitle
\providecommand{\subtitle}[1]{
  \posttitle{
    \begin{center}\large#1\end{center}
    }
}

\setlength{\droptitle}{-2em}

  \title{\Huge\textbf{COMP 4001 - Assignment C}}
    \pretitle{\vspace{\droptitle}\centering\huge}
  \posttitle{\par}
  \subtitle{Name: Tri Cao -- Student number: 100971065}
  \author{}
    \preauthor{}\postauthor{}
      \predate{\centering\large\emph}
  \postdate{\par}
    \date{01 February 2020}

\usepackage{float}
\let\origfigure\figure
\let\endorigfigure\endfigure
\renewenvironment{figure}[1][2] {
    \expandafter\origfigure\expandafter[H]
} {
    \endorigfigure
}

\begin{document}
\maketitle

\newgeometry{top=1in,bottom=1in,right=0.5in,left=1in}

\hypertarget{question-1}{%
\subsection{Question 1}\label{question-1}}

\hypertarget{question-2}{%
\subsection{Question 2}\label{question-2}}

\hypertarget{pseudo-code-of-the-flooding-algorithm-from-a-given-node}{%
\paragraph{Pseudo-code of the flooding algorithm from a given
node}\label{pseudo-code-of-the-flooding-algorithm-from-a-given-node}}

\hypertarget{question-3}{%
\subsection{Question 3}\label{question-3}}

\begin{itemize}
\tightlist
\item
  We prove the statement using contradiction. Let \(G\) be the graph
  whose edges have distinct weights, and \(G\) also have 2 different MST
  \(T\) and \(T'\). Because they are different, \(T\) and \(T'\) must
  contain different edges, therefore the set of all edges belong to
  \(T\) but not \(T'\), or belong to \(T'\) but not \(T\), must not be
  empty.
\item
  Consider the smallest edge \(e\) that belong to at most subgraph \(T\)
  or \(T'\). Without the loss of generality, We assume that
  \(e \in T \backslash T'\)
\item
  Adding this edge to \(T'\) will create a cycle (according to spanning
  tree definition). We consider 2 cases:

  \begin{itemize}
  \tightlist
  \item
    if all edges in the cycle in \(T'\) are in \(T\), then \(T\)
    contains a cycle, which contradicts the fact that \(T\) is a
    spanning tree
  \item
    There exists an edge \(e' \notin T\) in the cycle. Know that remove
    \(e'\) from the cycle results in a spanning tree. But since
    \(w(e') > w(e)\), since \(e\) is the smallest distinct edge and all
    edges in \(G\) have different weights, removing \(e'\) will create a
    smaller spanning tree. This contradict the fact that \(T'\) is the
    MST.
  \end{itemize}
\item
  Thus, by contradiction, we conclude that if all edges in a graph \(G\)
  have distinct weights, then there is exactly one MST for \(G\).
\end{itemize}

\hypertarget{question-4}{%
\subsection{Question 4}\label{question-4}}

\hypertarget{question-5}{%
\subsection{Question 5}\label{question-5}}

\hypertarget{question-6}{%
\subsection{Question 6}\label{question-6}}

\hypertarget{question-7}{%
\subsection{Question 7}\label{question-7}}

\hypertarget{question-8}{%
\subsection{Question 8}\label{question-8}}

\hypertarget{question-9}{%
\subsection{Question 9}\label{question-9}}

\hypertarget{question-10}{%
\subsection{Question 10}\label{question-10}}

\hypertarget{question-11}{%
\subsection{Question 11}\label{question-11}}


\end{document}
